%%%%%%%%%%%%%%%%%%%%%%%%%%%%%%%%%%%%%%%%%%%%%%%%%%%
%% LaTeX book template                           %%
%% Author:  Amber Jain (http://amberj.devio.us/) %%
%% License: ISC license                          %%
%%%%%%%%%%%%%%%%%%%%%%%%%%%%%%%%%%%%%%%%%%%%%%%%%%%

\documentclass[a4paper,11pt,oneside]{book}
\usepackage{../../modulestyle}

%%%%%%%%%%%%%%%%%%%%%%%%%%%%%%%%%%%%%%%%%%%%%%%%%%%%%%%%%
% Source: http://en.wikibooks.org/wiki/LaTeX/Hyperlinks %
%%%%%%%%%%%%%%%%%%%%%%%%%%%%%%%%%%%%%%%%%%%%%%%%%%%%%%%%%

%%%%%%%%%%%%%%%%%%%%%%%%%%%%%%%%%%%%%%%%%%%%%%%%%%%%%%%%%%%%%%%%%%%%%%%%%%%%%%%%
% 'dedication' environment: To add a dedication paragraph at the start of book %
% Source: http://www.tug.org/pipermail/texhax/2010-June/015184.html            %
%%%%%%%%%%%%%%%%%%%%%%%%%%%%%%%%%%%%%%%%%%%%%%%%%%%%%%%%%%%%%%%%%%%%%%%%%%%%%%%%
\newenvironment{dedication}
{
   \cleardoublepage
   \thispagestyle{empty}
   \vspace*{\stretch{1}}
   \hfill\begin{minipage}[t]{0.66\textwidth}
   \raggedright
}
{
   \end{minipage}
   \vspace*{\stretch{3}}
   \clearpage
}

%%%%%%%%%%%%%%%%%%%%%%%%%%%%%%%%%%%%%%%%%%%%%%%%
% Chapter quote at the start of chapter        %
% Source: http://tex.stackexchange.com/a/53380 %
%%%%%%%%%%%%%%%%%%%%%%%%%%%%%%%%%%%%%%%%%%%%%%%%
\makeatletter
\renewcommand{\@chapapp}{}% Not necessary...
\newenvironment{chapquote}[2][2em]
  {\setlength{\@tempdima}{#1}%
   \def\chapquote@author{#2}%
   \parshape 1 \@tempdima \dimexpr\textwidth-2\@tempdima\relax%
   \itshape}
  {\par\normalfont\hfill--\ \chapquote@author\hspace*{\@tempdima}\par\bigskip}
\makeatother

%%%%%%%%%%%%%%%%%%%%%%%%%%%%%%%%%%%%%%%%%%%%%%%%%%%
% First page of book which contains 'stuff' like: %
%  - Book title, subtitle                         %
%  - Book author name                             %
%%%%%%%%%%%%%%%%%%%%%%%%%%%%%%%%%%%%%%%%%%%%%%%%%%%

\newcommand{\CourseTitle}{Data Structures and Algorithms}
\newcommand{\ChapterNumber}{1}
\newcommand{\ChapterTitle}{Introduction to Data Structures and Algorithms}
\newcommand{\CodingNumber}{1}
\newcommand{\CodingTitle}{Introduction to Data Structures and Algorithms}
\newcommand{\SubmissionDeadline}{October 11, 2024}
\newcommand{\SubmissionDeadlineText}{on or before \SubmissionDeadline}
\newcommand{\SubmissionTemplateURL}{https://docs.google.com/document/d/1zZf3W0Hj6NfCGU7sKAaz_cztSER5e7Vx/edit?usp=sharing&ouid=112709378145681657270&rtpof=true&sd=true}

\newcommand{\BookTitle}{Coding Exercise: \CodingNumber - \CodingTitle}
\newcommand{\BookTitleFootnote}{A coding exercise for
\textnormal{Chapter \ChapterNumber} of the Study Guide on the course \CourseTitle.}

\newcommand{\BookSubtitle}{Chapter \ChapterNumber: \ChapterTitle}
\newcommand{\BookSubtitleFootnote}{This chapter introduces the basic concepts of data structures and algorithms.}

\newcommand{\BookAuthorFirstName}{Jarrian Vince}
\newcommand{\BookAuthorLastName}{Gojar}
\newcommand{\BookAuthorName}{Jarrian Vince G. Gojar}
\newcommand{\BookAuthorURL}{https://github.com/godkingjay}

\newcommand{\GoogleDriveURLBSITTwoFour}{https://drive.google.com/drive/folders/1uc3ehhK4Mv84KXPe8oV3l3AGI6czVhBx?usp=sharing}
\newcommand{\GoogleDriveURLBSITTwoFive}{https://drive.google.com/drive/folders/1eIkUp3t2cAKIpd9KZGbQlZEGU516S_sE?usp=sharing}

\newcommand{\FolderFormat}{Group Number - LastName1\_FirstName1, LastName2\_FirstName2}
\newcommand{\FolderFormatExample}{Group 1 - Doe\_John, Smith\_Jane}

% Book's title and subtitle
\title{\Huge \textbf{\BookTitle}  \footnote{\BookTitleFootnote} \\
\huge \BookSubtitle \footnote{\BookSubtitleFootnote}}

% Author
\author{\textsc{\BookAuthorName}\thanks{\url{\BookAuthorURL}}}

\begin{document}

\frontmatter
\date{}
\maketitle

%%%%%%%%%%%%%%%%%%%%%%%%%%%%%%%%%%%%%%%%%%%%%%%%%%%%%%%%%%%%%%%
% Add a dedication paragraph to dedicate your book to someone %
%%%%%%%%%%%%%%%%%%%%%%%%%%%%%%%%%%%%%%%%%%%%%%%%%%%%%%%%%%%%%%%
\begin{dedication}
Sorsogon State University - Bulan Campus
\end{dedication}

%%%%%%%%%%%%%%%%%%%%%%%%%%%%%%%%%%%%%%%%%%%%%%%%%%%%%%%%%%%%%%%%%%%%%%%%
% Auto-generated table of contents, list of figures and list of tables %
%%%%%%%%%%%%%%%%%%%%%%%%%%%%%%%%%%%%%%%%%%%%%%%%%%%%%%%%%%%%%%%%%%%%%%%%

\mainmatter

%%%%%%%%%%%
% Preface %
%%%%%%%%%%%
\section*{For Loop}

In C++, the \texttt{for} loop is used to execute a block of code repeatedly.
It is often used when the number of iterations is known. The syntax of the
\texttt{for} loop is as follows:

\begin{lstlisting}[language=C++, caption={Syntax of the for loop}]
for (initialization; condition; advancement) {
    // code block to be executed
}
\end{lstlisting}

\begin{itemize}
    \item \textbf{Initialization:} It is executed only once when the loop
    starts.
    \item \textbf{Condition:} It is evaluated before each iteration. If it
    is true, the code block is executed.
    \item \textbf{Advancement:} It is executed after the code block has
    been executed.
\end{itemize}
  
The following example demonstrates the use of the \texttt{for} loop to print
the elements of an array.

\begin{lstlisting}[language=C++, caption={Example of the for loop}]
#include <iostream>
using namespace std;

int main() {
    int arr[5] = {1, 2, 3, 4, 5};
    int n = sizeof(arr) / sizeof(arr[0]);

    for (int i = 0; i < n; i++) {
        cout << arr[i] << " ";
    }

    return 0;
}
\end{lstlisting}

The code above shows a simple program that prints the elements of an array.
It initializes an array of integers with five elements and then uses a
\texttt{for} loop to print the elements to the console. The `n' variable
stores the size of the array, which is used as the condition for the loop.

The output of the program is as follows:

\begin{lstlisting}[language=bash, caption={Output of the for loop}]
1 2 3 4 5
\end{lstlisting}

\section*{Coding Exercises}

\begin{chapquote}{Bjarne Stroustrup}
``C makes it easy to shoot yourself in the foot; C++ makes it harder, but when
you do, it blows away your whole leg.''
\end{chapquote}

Instructions: Write a program that solves the following problems.
Submit your code to the Google Drive folder provided by the instructor.

% The following are the exercises

\begin{enumerate}
  \item Implement a C++ program that demonstrates the primitive data types.
  \begin{enumerate}
      \item Declare and initialize variables of the following different data types.
      \begin{enumerate}
          \item Integer
          \item Float
          \item Double
          \item Character
          \item Boolean
      \end{enumerate}
      \item Print the values of the variables to the console.
  \end{enumerate}
  \item Implement a C++ program to find the maximum element in an array
  using linear time complexity.
  \begin{enumerate}
      \item Declare an array of integers.
      \begin{align*}
          \text{int arr[6];}
      \end{align*}
      \item Initialize the array with random values.
      \begin{align*}
          \text{arr[6] = \{19, 10, 8, 17, 9, 15\};}
      \end{align*}
      \item Find the maximum element in the array.
      \item Print the maximum element to the console.
      \begin{quote}
          Output: 19
      \end{quote}
      \item Determine the \textbf{time complexity} and \textbf{space complexity} of the program.
  \end{enumerate}
  \item Implement a C++ program to find the sum of all elements in an array
  using linear time complexity.
  \begin{enumerate}
      \item Declare an array of integers.
      \begin{align*}
          \text{int arr[6];}
      \end{align*}
      \item Initialize the array with random values.
      \begin{align*}
          \text{arr[6] = \{19, 10, 8, 17, 9, 15\};}
      \end{align*}
      \item Find the sum of all elements in the array.
      \item Print the sum to the console.
      \begin{quote}
          Output: 78
      \end{quote}
      \item Determine the \textbf{time complexity} and \textbf{space complexity} of the program.
  \end{enumerate}
  % \item Implement a C++ program to reverse an array in place using linear
  % time complexity.
  % \item Implement a C++ program to sort an array using the bubble sort
  % algorithm with quadratic time complexity.
  % \item Implement a C++ program to generate all subsets of a set of elements
  % using exponential space complexity.
  % \item Implement a C++ program to generate all permutations of a set of
  % elements using factorial space complexity.
\end{enumerate}

\section*{Submission of Coding Exercises}

Instructions:
\begin{enumerate}
  \item Go to the Google Drive folder provided by the instructor: \\
  \begin{quote}
    \textbf{BSIT 2-4:} \\ \url{\GoogleDriveURLBSITTwoFour} \\ \\
    \textbf{BSIT 2-5:} \\ \url{\GoogleDriveURLBSITTwoFive}
  \end{quote}
  \item Inside the folder, create another folder for your group
  with the following format:
    \begin{quote}
      \textbf{\FolderFormat} \\
      Example: \textbf{\FolderFormatExample}
    \end{quote}
  \item Inside the sub-folder, create another folder with the name:
    \begin{quote}
        \textbf{Chapter \ChapterNumber - Coding Exercise \CodingNumber - \CodingTitle}
    \end{quote}
  \item Inside the folder, upload the file of your submission.
    \begin{quote}
        Fill in the template provided in the following link and upload
        it inside the folder: \\
        \url{\SubmissionTemplateURL}
    \end{quote}
  \item The activity must be submitted \textbf{\SubmissionDeadlineText}.
  \item Late submissions will not be accepted.
\end{enumerate}

\end{document}